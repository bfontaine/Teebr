%!TEX root = rapport.tex

\section{Conclusion}

Ce projet est ambitieux et a rencontré plus de difficultés que
prévu\footnote{Est-ce vraiment une surprise ? :)}. La taille des documents
utilisés ainsi le problème de qualité de la source de contenu a compliqué la
tâche, mais les résultats montrent que la méthodologie utilisée ici n’est pas
suffisante pour faire de la recommandation pertinente. Ce projet se veut
expérimental, et son but était de pouvoir répondre à la question suivante :
est-il possible de faire de la recommandation sur des textes très courts sans
jamais utiliser le graphe social, ce qui semble pourtant être la solution la
plus simple ici ? Au vu des résultats du projet, il nous semble que son
objectif a été atteint, et que la réponse est « non. » Nous aurions pu explorer
d’autres pistes (\ref{sec:pistes}) qui auraient amélioré les résultats mais le
ratio résultat/effort semble être plus important en utilisant le graphe social.

\section{Ressources}

\begin{itemize}
  \item Sources du projet : \url{https://github.com/teebr/Teebr}
  \item Démo en ligne : \url{http://teebr.co}
\end{itemize}

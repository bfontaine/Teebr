%!TEX root = rapport.tex

\section{Introduction}

Les moteurs de recommandation de contenu cherchent à trouver du contenu
pertinent en fonction de sa similarité avec du contenu consommé par un
utilisateur, tandis que les moteurs de recommandation de personnes sur les
réseaux sociaux (par exemple ami(e)s ou collègues) se basent principalement sur
le graph social et la place de chaque utilisateur dans celui-ci.

On s’intéresse ici au réseau social \twt{}, où chaque utilisateur peut produire
du contenu sous forme de courts messages (\tweets\footnote{On utilise ici les
  recommandations en matière de traduction de \twt{}, qui veut, entre autres,
  que l’on capitalise le mot \tweet{}.}), lisibles par tous sur son
profil. Un utilisateur peut \I{s’abonner} à d’autres utilisateurs, ce qui lui
permet de recevoir le flux continu de tous les messages de ces abonnements sans
avoir à consulter chaque profil individuellement. Cette notion d’abonnement est
asymétrique : l’abonnement de A à B n’implique pas l’abonnement de B à A. Dans
le vocabulaire de la plateforme, on parle de \I{suivre} un utilisateur.

\twt{} est intéressant ici pour plusieurs points : le premier est la longueur
des messages. Ceux-ci sont très courts (140 caractères maximum), ce qui rend
l’extraction d’information plus difficile, car il n’y a pas de redondance et
beaucoup de termes sont abrégés. Le second est l’accès aux messages, qui sont
tous publics.\footnote{Il est possible de rendre un compte privé, mais peu
d’utilisateurs le font.} L’\api{} de \twt{} est par ailleurs très bien faite,
et il est facile de récupérer des messages, que l’on peut filtrer selon
plusieurs critères. Cela n’aurait pas été possible avec d’autres réseaux comme
\fb{}. Enfin, la qualité du flux que reçoit un utilisateur dépend des personnes
auxquelles il est abonné, donc la recommandation de contenu est fortement liée
à la recommandation d’utilisateurs.

Le projet \tb{} a pour but de faire de la recommandation d’utilisateurs sur
\twt{} non pas en fonction de leur place dans le graph social comme on le fait
habituellement, mais en fonction du contenu qu’ils produisent. On utilisera
pour cela un système de notation de \tweets{} pour construire le profil d’un
utilisateur.

\section{Définitions}

Nous définissons ici un certain nombre de termes utilisés dans la suite de ce
rapport.

\subsection{Twitter}

\twt{} est un réseau social fondé il y a bientôt dix ans (2006) qui possède son
propre vocabulaire. Les \I{\tweets{}} sont les courts messages postés par les
utilisateurs. Un \I{\rt{}} est la retransmission d’un \tweet{} d’un utilisateur
par un autre. \I{Retweeter} un message permet à un utilisateur de le partager
avec l’ensemble de ses abonnés. Une \I{réponse} est un message posté pour
\I{répondre} à un autre. Les utilisateurs de \twt{} peuvent ainsi s’engager
dans des conversations en \I{répondant} aux \tweets{} d’autres utilisateurs.

\subsection{Producteurs et consommateurs}

\tb{} est conçu pour fonctionner avec plusieurs utilisateurs. Ceux-ci sont
appelés \I{Consommateurs} car ils \I{consomment} du contenu en le lisant. Ils
sont l’opposé des \I{Producteurs}, qui sont les comptes \twt{} dont on récupère
le contenu (les \tweets{}). Leur rôle est inversé entre \twt{} et la plateforme
\tb{} : sur le premier, les producteurs sont actifs tandis que les
consommateurs sont passifs. Sur \tb{}, les consommateurs sont actifs (ils
notent du contenu) tandis que les producteurs sont passifs. On désigne ici par
« contenu » les \I{\tweets{}} produits par un utilisateur (un producteur,
donc).

\subsection{Signatures}

Chaque consommateur, producteur et \tweet{} possède une \I{Signature}. Celle-ci
est un vecteur d’entiers, chaque entier correspondant à une \I{Caractéristique}
(\I{feature}). Ces caractéristiques sont les mêmes pour les trois entités. La
signature d’un \tweet{} est fonction de ses caractéristiques propres (langue,
source, présence ou non de lien, sujet, etc). La signature d’un producteur est
un produit de l’ensemble des signatures des \tweets{} qu’il a écrit. La
signature d’un consommateur est un produit de l’ensemble des signatures des
\tweets{} qu’il a noté.
